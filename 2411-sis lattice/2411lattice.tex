%%
% Please see https://bitbucket.org/rivanvx/beamer/wiki/Home for obtaining beamer.
%%
\documentclass{beamer}
\usetheme{Boadilla}
\usecolortheme{beaver}
\usepackage{amsmath,amscd}
\usepackage{multicol}

\title{LWE/(SIS)}
\subtitle{and how they are related with lattice}
\author{Kurt Pan}
\institute{\url{ZKPunk.pro}}
\date{\today}

%\logo{\includegraphics[height=1.5cm]{zkpunk-qrcode.png}

\AtBeginSection[]
{
  \begin{frame}
    \frametitle{Table of Contents}
    \tableofcontents[currentsection]
  \end{frame}
}
\setbeamercolor{section number projected}{bg=red}

\begin{document}

\begin{frame}
	\titlepage 
	\begin{columns}
	\column{0.5\textwidth}
	\centering
	\includegraphics[width=3cm]{zkpunk-qrcode.png}
    \column{0.5\textwidth}
    \centering
    \includegraphics[width=3cm]{lazarus-repo.png}
 \end{columns}

\end{frame}
  
      	
  

\begin{frame}
\frametitle{Table of Contents}
\tableofcontents
\end{frame}

\section{\textcolor{red}{What is LWE}}

\begin{frame}
	\frametitle{What is Learning with Errors}
	\begin{definition}[LWE Distribution]
		Let $\phi$ be a probability density function on $\mathbb{T}$, and $\mathbf{s} \in \mathbb{Z}_p^n$ denote the unknown secret vector. The LWE distribution $A_{\mathbf{s}, \phi}$ is the distribution over $\mathbb{Z}_p^n \times \mathbb{T}$ obtained by choosing $\mathbf{a} \in \mathbb{Z}_p^n$ uniformly at random and $e \in \mathbb{T}$ according to $\phi$, then outputting ( $\left.\mathbf{a}, \frac{1}{p}\langle\mathbf{a}, \mathbf{s}\rangle+e\right)$.
	\end{definition}

\end{frame}

\begin{frame}
	\frametitle{}

	\begin{definition}[Decision-LWE]
	 Given a polynomial number of samples either from the distribution $A_{\mathbf{s}, \phi}$ or independent and uniformly distributed samples from $\mathbb{Z}_p^n \times \mathbb{T}$, output
	 \begin{itemize}
	 	\item YES if the samples are from the LWE distribution $A_{\mathbf{s}, \phi}$, or
	 	\item NO if the samples are uniformly random over $\mathbb{Z}_p^n \times \mathbb{T}$.
	 \end{itemize}
	\end{definition}
	\includegraphics[scale=0.5]{lwe.png}
\end{frame}



\section{\textcolor{red}{What is Lattice}}


\begin{frame}
	\frametitle{What is Lattice}
	\begin{definition}[Lattice]
		A discrete additive subgroup of $\mathbb{R}^n$
	\end{definition}
	A lattice is the set of all \emph{integer} linear combinations of (linearly independent) \emph{basis} vectors $\mathbf{B}=\left\{\mathbf{b}_1, \ldots, \mathbf{b}_n\right\} \subset \mathbb{R}^n$ :

$$
\mathcal{L}=\sum_{i=1}^n \mathbf{b}_i \cdot \mathbb{Z}=\left\{\mathbf{B} \mathbf{x}: \mathbf{x} \in \mathbb{Z}^n\right\}
$$

The same lattice has many bases

\end{frame}


\begin{frame}
	\frametitle{Shortest Vector Problem}
	\begin{definition}[SVP$_\gamma$]
		Given a lattice $\mathcal{L}(\mathbf{B})$, find a (nonzero) lattice vector $\mathbf{B x}$ (with $\mathbf{x} \in \mathbb{Z}^k$ ) of length (at most) $\|\mathbf{B} \mathbf{x}\| \leq \gamma \lambda_1$
	\end{definition}
	\includegraphics[scale=0.5]{svp.png}
	
	Minimum distance

$$
\begin{aligned}
\lambda_1 & =\min _{\mathbf{x}, \mathbf{y} \in \mathcal{L}, \mathbf{x} \neq \mathbf{y}}\|\mathbf{x}-\mathbf{y}\| \\
& =\min _{\mathbf{x} \in \mathcal{L}, \mathbf{x} \neq \mathbf{0}}\|\mathbf{x}\|
\end{aligned}
$$


\end{frame}

\begin{frame}
	\frametitle{Shortest Independent Vectors Problem}
	\begin{definition}[SIVP$_\gamma$]
		Given a lattice $\mathcal{L}(\mathbf{B})$, find $n$ linearly independent lattice vectors $\mathbf{B x}_1, \ldots, \mathbf{B} \mathbf{x}_n$ of length (at most) $\max _i\left\|\mathbf{B} \mathbf{x}_i\right\| \leq \gamma \lambda_n$
	\end{definition}
	\includegraphics[scale=0.7]{sivp.png}

\end{frame}

\begin{frame}
	\frametitle{Closest Vector Problem}
	\begin{definition}[CVP$_\gamma$]
		Given a lattice $\mathcal{L}(\mathbf{B})$ and a target point $\mathbf{t}$, find a lattice vector $\mathbf{B x}$ within distance $\|\mathbf{B} \mathbf{x}-\mathbf{t}\| \leq \gamma \mu$ from the target
		
	\end{definition}
	
	\includegraphics[scale=0.7]{cvp.png}
	

\end{frame}



\begin{frame}
	\frametitle{Special Versions of CVP}

	\begin{definition}
		Given $(\mathcal{L}, \mathbf{t}, \boldsymbol{d})$, with $\mu(\mathbf{t}, \mathcal{L}) \leq \boldsymbol{d}$, find a lattice point within distance $d$ from $\boldsymbol{t}$.
	\end{definition}
	
	\begin{itemize}
		\item If $d$ is arbitrary, then one can find the closest lattice vector by binary search on $d$.
		\item \emph{Bounded Distance Decoding (BDD)}: If $d<\lambda_1(\mathcal{L}) / 2$, then there is at most one solution. Solution is the closest lattice vector.
		\item \emph{Absolute Distance Decoding (ADD)}: If $d \geq \mu(\mathcal{L})$, then there is always at least one solution. Solution may not be closest lattice vector.


	\end{itemize}
\end{frame}



\begin{frame}
	\frametitle{Relations among lattice problems}
	\includegraphics[scale=0.6]{latticeproblem.png}

\end{frame}




\section{\textcolor{red}{How are they related?}}

\begin{frame}
	\frametitle{Random lattices in Cryptography}
	\begin{itemize}
		\item Cryptography typically uses (random) lattices $\wedge$ such that
 $\wedge \subseteq \mathbb{Z}^d$ is an integer lattice
and $q \mathbb{Z}^d \subseteq \Lambda$ is periodic modulo a small integer $q$.
\item Cryptographic functions based on $q$-ary lattices involve only arithmetic modulo $q$.
	\end{itemize}
	
	\begin{definition}[$q$-ary lattice]
$\Lambda$ is a $q$-ary lattice if $q \mathbb{Z}^n \subseteq \Lambda \subseteq \mathbb{Z}^n$
	\end{definition}
	Examples (for any $\mathbf{A} \in \mathbb{Z}_q^{n \times d}$ )
	\begin{itemize}
		\item $\Lambda_q(\mathbf{A})=\left\{\mathbf{x} \mid \mathbf{x} \bmod q \in \mathbf{A}^T \mathbb{Z}_q^n\right\} \subseteq \mathbb{Z}^d$
		\item $\Lambda_q^{\perp}(\mathbf{A})=\{\mathbf{x} \mid \mathbf{A x}=\mathbf{0} \bmod q\} \subseteq \mathbb{Z}^d$
	\end{itemize}
	
	$$
\mathcal{L}_q^{\perp}\left(\left[\mathbf{A} \mid \mathbf{I}_n\right]\right)=\mathcal{L}\left(\left[\begin{array}{cc}
-\mathbf{I}_m & \mathbf{0} \\
\mathbf{A} & q \mathbf{I}_n
\end{array}\right]\right)
$$
\end{frame}

\begin{frame}
	\frametitle{}
	$$
\begin{aligned}
& \mathbf{A} \in \mathbb{Z}_q^{m \times k}, \mathbf{s} \in \mathbb{Z}_q^k, \mathbf{e} \in \mathcal{E}^m . \\
& g_{\mathbf{A}}(\mathbf{s} ; \mathbf{e})=\mathbf{A} \mathbf{s}+\mathbf{e} \bmod q
\end{aligned}
$$
	\begin{theorem} [R'05]
The function $g_{\mathrm{A}}(\mathbf{s}, \mathbf{e})$ is hard to invert on the average, assuming SIVP is hard to approximate in the worst-case. 
	\end{theorem}

\end{frame}


\begin{frame}
	\frametitle{How are they related?}
	LWE and q-ary lattices
	\begin{itemize}
		\item If $\mathbf{e}=\mathbf{0}$, then $\mathbf{A s}+\mathbf{e}=\mathbf{A} \mathbf{s} \in \Lambda\left(\mathbf{A}^t\right)$
		\item Same as CVP in random $q$-ary lattice $\Lambda\left(\mathbf{A}^t\right)$ with random target $\mathbf{t}=\mathbf{A s + e}$
		\item Usually $\mathbf{e}$ is shorter than $\frac{1}{2} \lambda_1\left(\Lambda\left(\mathbf{A}^T\right)\right)$, and $\mathbf{e}$ is uniquely determined
		\item TAKE AWAY: 
		\item LWE $\equiv$ Approximate BDD (Bounded Distance Decoding)
	\end{itemize}

	
\end{frame}


\begin{frame}
	\frametitle{SIS for next time!}
	\includegraphics[scale=0.7]{lwesis.png}
\end{frame}



\begin{frame}
	\centering \Large{Thanks! Kob-khun kub!}
\end{frame}

\end{document}
