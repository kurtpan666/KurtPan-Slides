%%
% Please see https://bitbucket.org/rivanvx/beamer/wiki/Home for obtaining beamer.
%%
\documentclass{beamer}
\usetheme{Boadilla}
\usecolortheme{beaver}
\usepackage{amsmath,amscd}
\usepackage{multicol}



\title{Road to Beam Chain}
\subtitle{A lattice approach}
\author{Kurt Pan}
\institute{\url{ZKPunk.pro}}
\date{\today}

%\logo{\includegraphics[height=1.5cm]{zkpunk-qrcode.png}

\AtBeginSection[]
{
  \begin{frame}
    \frametitle{Table of Contents}
    \tableofcontents[currentsection]
  \end{frame}
}
\setbeamercolor{section number projected}{bg=red}

\begin{document}

\begin{frame}
	\titlepage 
	\begin{columns}
	\column{0.5\textwidth}
	\centering
	\includegraphics[width=3cm]{zkpunk-qrcode.png}
    \column{0.5\textwidth}
    \centering
    \includegraphics[width=3cm]{lazarus-repo.png}
 \end{columns}
\end{frame}

\section{Why Post-Quantum}

\begin{frame}
\frametitle{Beam Chain in a nutshell}
	\includegraphics[scale=0.25]{beamzk.jpeg}
	
\end{frame}

\begin{frame}
\frametitle{Beam Chain Roadmap}
\includegraphics[scale=0.25]{beamroad.webp}

	faster block times, faster finality,
	
	 \textcolor{red}{chain snarkification, and quantum resistance.}
\end{frame}

\begin{frame}
\frametitle{Disallowed Pre-Quantum Signature Schemes after 2035}
\includegraphics[scale=0.24]{nistpqc.png}
\end{frame}

\begin{frame}
	\includegraphics[scale=0.5]{pqiq.jpg}

\end{frame}



\section{Why Lattice}



\begin{frame}
	\frametitle{Why NOT Lattice?}
	\includegraphics[scale=0.25]{danboneh.png}
	Wechat: XPTY
	\includegraphics[scale=0.35]{justin-lattice.jpg}
\end{frame}

\begin{frame}
\frametitle{FIP 206 : FN-DSA (planned for late 2024)}
\includegraphics[scale=0.3]{fips.png} 
\includegraphics[scale=0.3]{fips2.png}

\footnote{\url{https://www.nist.gov/news-events/news/2024/08/nist-releases-first-3-finalized-post-quantum-encryption-standards}}
\end{frame}

\begin{frame}
\begin{itemize}
	\item Compared to other signature schemes selected for standardisation by NIST, such as Dilithium [LDK+22] and Sphincs+ [HBD+22], 
	\item Falcon stands out for its compactness, minimising both public key and signature sizes. \footnote{\url{https://falcon-sign.info/}}
\end{itemize}

 \centering
  \resizebox{\textwidth}{!}{%
    \begin{tabular}{|l|r|r|r|r|r|r|}
      \hline
      variant & keygen (ms) & keygen (RAM) & sign/s & verify/s & pub size & sig size \\
      \hline
      FALCON-512 & 8.64 & 14336 & 5948.1 & 27933.0 & 897 & 666 \\
      \hline
      FALCON-1024 & 27.45 & 28672 & 2913.0 & 13650.0 & 1793 & 1280 \\
      \hline
    \end{tabular}%
  }

\includegraphics[scale=0.3]{slhdsa.png}
\end{frame}

\section{What is lattice}

\begin{frame}
	\frametitle{What is Lattice}
	\begin{definition}[Lattice]
		A discrete additive subgroup of $\mathbb{R}^n$
		
			A lattice is the set of all \emph{integer} linear combinations of (linearly independent) \emph{basis} vectors $\mathbf{B}=\left\{\mathbf{b}_1, \ldots, \mathbf{b}_n\right\} \subset \mathbb{R}^n$ :

$$
\mathcal{L}=\sum_{i=1}^n \mathbf{b}_i \cdot \mathbb{Z}=\left\{\mathbf{B} \mathbf{x}: \mathbf{x} \in \mathbb{Z}^n\right\}
$$
	\end{definition}


The same lattice has (infinite) many bases.

\centering
\includegraphics[scale=0.2]{lattice.png}
\end{frame}


\begin{frame}
	\frametitle{Shortest Vector Problem}
	\begin{definition}[SVP$_\gamma$]
		Given a lattice $\mathcal{L}(\mathbf{B})$, find a (nonzero) lattice vector $\mathbf{B x}$ (with $\mathbf{x} \in \mathbb{Z}^k$ ) of length (at most) $\|\mathbf{B} \mathbf{x}\| \leq \gamma \lambda_1$
	\end{definition}
	\includegraphics[scale=0.5]{svp.png}
	
	Minimum distance

$$
\begin{aligned}
\lambda_1 & =\min _{\mathbf{x}, \mathbf{y} \in \mathcal{L}, \mathbf{x} \neq \mathbf{y}}\|\mathbf{x}-\mathbf{y}\| \\
& =\min _{\mathbf{x} \in \mathcal{L}, \mathbf{x} \neq \mathbf{0}}\|\mathbf{x}\|
\end{aligned}
$$


\end{frame}

\begin{frame}
	\frametitle{Shortest Independent Vectors Problem}
	\begin{definition}[SIVP$_\gamma$]
		Given a lattice $\mathcal{L}(\mathbf{B})$, find $n$ linearly independent lattice vectors $\mathbf{B x}_1, \ldots, \mathbf{B} \mathbf{x}_n$ of length (at most) $\max _i\left\|\mathbf{B} \mathbf{x}_i\right\| \leq \gamma \lambda_n$
	\end{definition}
	\includegraphics[scale=0.7]{sivp.png}

\end{frame}

\begin{frame}
	\frametitle{Closest Vector Problem}
	\begin{definition}[CVP$_\gamma$]
		Given a lattice $\mathcal{L}(\mathbf{B})$ and a target point $\mathbf{t}$, find a lattice vector $\mathbf{B x}$ within distance $\|\mathbf{B} \mathbf{x}-\mathbf{t}\| \leq \gamma \mu$ from the target
		
	\end{definition}
	
	\includegraphics[scale=0.7]{cvp.png}
	

\end{frame}



\begin{frame}
	\frametitle{Special Versions of CVP}

	\begin{definition}
		Given $(\mathcal{L}, \mathbf{t}, \boldsymbol{d})$, with $\mu(\mathbf{t}, \mathcal{L}) \leq \boldsymbol{d}$, find a lattice point within distance $d$ from $\boldsymbol{t}$.
	\end{definition}
	
	\begin{itemize}
		\item If $d$ is arbitrary, then one can find the closest lattice vector by binary search on $d$.
		\item \emph{Bounded Distance Decoding (BDD)}: If $d<\lambda_1(\mathcal{L}) / 2$, then there is at most one solution. Solution is the closest lattice vector.
		\item \emph{Absolute Distance Decoding (ADD)}: If $d \geq \mu(\mathcal{L})$, then there is always at least one solution. Solution may not be closest lattice vector.


	\end{itemize}
\end{frame}



\begin{frame}
	\frametitle{Relations among lattice problems}
	\includegraphics[scale=0.6]{latticeproblem.png}

\end{frame}


\begin{frame}
	\frametitle{Random lattices in Cryptography}
	\begin{itemize}
		\item Cryptography typically uses (random) lattices $\wedge$ such that
 $\wedge \subseteq \mathbb{Z}^d$ is an integer lattice
and $q \mathbb{Z}^d \subseteq \Lambda$ is periodic modulo a small integer $q$.
\item Cryptographic functions based on $q$-ary lattices involve only arithmetic modulo $q$.
	\end{itemize}
	
	\begin{definition}[$q$-ary lattice]
$\Lambda$ is a $q$-ary lattice if $q \mathbb{Z}^n \subseteq \Lambda \subseteq \mathbb{Z}^n$
	\end{definition}
	Examples (for any $\mathbf{A} \in \mathbb{Z}_q^{n \times d}$ )
	\begin{itemize}
		\item $\Lambda_q(\mathbf{A})=\left\{\mathbf{x} \mid \mathbf{x} \bmod q \in \mathbf{A}^T \mathbb{Z}_q^n\right\} \subseteq \mathbb{Z}^d$
		\item $\Lambda_q^{\perp}(\mathbf{A})=\{\mathbf{x} \mid \mathbf{A x}=\mathbf{0} \bmod q\} \subseteq \mathbb{Z}^d$ (SIS lattice)
	\end{itemize}
	
	$$
\mathcal{L}_q^{\perp}\left(\left[\mathbf{A} \mid \mathbf{I}_n\right]\right)=\mathcal{L}\left(\left[\begin{array}{cc}
-\mathbf{I}_m & \mathbf{0} \\
\mathbf{A} & q \mathbf{I}_n
\end{array}\right]\right)
$$
\end{frame}

\begin{frame}
	\frametitle{How are they related?}
	$$
\begin{aligned}
& \mathbf{A} \in \mathbb{Z}_q^{m \times k}, \mathbf{s} \in \mathbb{Z}_q^k, \mathbf{e} \in \mathcal{E}^m . \\
& g_{\mathbf{A}}(\mathbf{s} ; \mathbf{e})=\mathbf{A} \mathbf{s}+\mathbf{e} \bmod q
\end{aligned}
$$
	\begin{theorem} [R'05]
The function $g_{\mathrm{A}}(\mathbf{s}, \mathbf{e})$ is hard to invert on the average, assuming SIVP is hard to approximate in the worst-case. 
	\end{theorem}
	
		
	LWE and q-ary lattices
	\begin{itemize}
		\item If $\mathbf{e}=\mathbf{0}$, then $\mathbf{A s}+\mathbf{e}=\mathbf{A} \mathbf{s} \in \Lambda\left(\mathbf{A}^t\right)$
		\item Same as CVP in random $q$-ary lattice $\Lambda\left(\mathbf{A}^t\right)$ with random target $\mathbf{t}=\mathbf{A s + e}$
		\item Usually $\mathbf{e}$ is shorter than $\frac{1}{2} \lambda_1\left(\Lambda\left(\mathbf{A}^T\right)\right)$, and $\mathbf{e}$ is uniquely determined
		\item TAKE AWAY: 
		\item LWE $\equiv$ Approximate BDD (Bounded Distance Decoding)
	\end{itemize}

\end{frame}



\begin{frame}
\frametitle{(M)SIS Problem}

\begin{definition}
	Let $n, m, \beta \in \mathbb{N}$. The Module Short Integer Solution problem M-SIS $_{n, m, \beta}$ over $\mathcal{R}_q$ is defined as follows. 
	
	Given $\boldsymbol{A} \stackrel{\$}{\leftarrow} \mathcal{R}_q^{n \times m}$, find an $\overrightarrow{\boldsymbol{z}} \in \mathcal{R}_q^m$ such that $\boldsymbol{A} \overrightarrow{\boldsymbol{z}}=\overrightarrow{\mathbf{0}}$ and $0<\|\overrightarrow{\boldsymbol{z}}\|_2 \leq \beta$.
\end{definition}

\includegraphics[scale=0.6]{sisfig.png}

\end{frame}

\begin{frame}
\frametitle{SIS (lattice formulation)}
\begin{definition}[SVP in SIS lattice]
	 Given $A \in_R \mathbb{Z}_q^{n \times m}$, find a nonzero $z \in[-B, B]^m$ in the SIS lattice $L_A^{\perp}=L(C)$ where $C=\left[\begin{array}{cc}q I_n & -\bar{A} \\ 0 & I_{m-n}\end{array}\right]$.
\end{definition}
	\includegraphics[scale=0.6]{lwesishard.png}
\end{frame}

\begin{frame}
	\frametitle{SIS application: Collision-resistant hash function (Ajtai)}
	
\begin{definition}[Ajtai Hash]
Select $A \in_R \mathbb{Z}_q^{n \times m}$, where $m>n \log q$.

Define $H_A:\{0,1\}^m \longrightarrow \mathbb{Z}_q^n$ by $H_A(z)=A z (\bmod q)$.
	
\end{definition}

\begin{proof}[Compression.]
	 Since $m>n \log q$, we have $2^m>q^n$.
\end{proof}

\begin{proof}[Collision resistance.]
	 Suppose that one can efficiently find $z_1, z_2 \in\{0,1\}^m$ with $z_1 \neq z_2$ and $H_A\left(z_1\right)=H_A\left(z_2\right)$. Then $A z_1=A z_2(\bmod q)$, whence $A z=0(\bmod q)$ where $z=z_1-z_2$. Since $z \neq 0$ and $z \in[-1,1]^m, z$ is an SIS solution (with $B=1$ ) which has been efficiently found.
\end{proof}

\end{frame}

\begin{frame}
\frametitle{NTRU: Nth-degree TRUncated polynomial ring}

\begin{itemize}
	\item The NTRU cryptosystem is parameterized by a certain polynomial ring $R=\mathbb{Z}[X] /(f(X))$, e.g., $f(X)=X^n-1$ for a prime $n$ or $f(X)=X^n+1$ for an $n$ that is a power of two, and a sufficiently large odd modulus $q$ that defines the quotient ring $R_q=R / q R$.
	\item The public key is $h=2 g \cdot s^{-1} \in R_q$ for two "short" polynomials $g, s \in R$, i.e., ones having relatively small integer coefficients, where the secret key $s$ is also chosen to be invertible modulo both $q$ and two. 
	\item Can define SVP/CVP in NTRU lattice
\end{itemize}

\end{frame}



\section{How to aggregate Falcon using Labrador}

\begin{frame}
\frametitle{Full-Domain-Hash (FDH) paradigm, hash-and-sign framework}


\begin{itemize}
	\item The verification key pk is a \emph{trapdoor permutation} $f$. 
	\item The signing key sk is the inverse $f^{-1}$. 
	\item To sign a message $m$, one first hashes $m$ to some point $y=\mathrm{H}(m)$ in the range of $f$, then outputs the signature $\sigma=f^{-1}(y)$. 
	\item Verification consists of checking that $f(\sigma)=\mathrm{H}(m)$.
\end{itemize}

\includegraphics[scale=0.4]{hashandsign.png}

Recall RSA-FDH / BLS signature	

\end{frame}

\begin{frame}
\frametitle{GPV framework: lattice-based hash-and-sign }
\begin{itemize}
	\item Generalised the FDH paradigm to work with \emph{preimage sampleable trapdoor functions (PSFs)}. 
	\item pk defining a PSF $F_{\mathrm{pk}}$, and sk that allows one to invert $F_{\mathrm{pk}}$.
	\item $y=\mathrm{H}(r, m) \in \mathrm{Ra}$, 
	\item Then use sk to sample a signature $\sigma=x \in \mathrm{Do}$ following some distribution $\mathcal{D}(F_{\mathrm{pk}}^{-1}(y))$. 
\end{itemize}
\end{frame}

\begin{frame}
\frametitle{Falcon: GPV over NTRU lattice}
\begin{itemize}
	\item  $\mathrm{pk}=\boldsymbol{h}=g \cdot f^{-1} \bmod q \in \mathcal{R}_q$ defines an NTRU-module $\Lambda=\left\{(\boldsymbol{u}, \boldsymbol{v}) \in \mathcal{R}^2: \boldsymbol{u}+\boldsymbol{h} \boldsymbol{v}=\mathbf{0} \bmod q\right\}$ 
	\item sk contains a secret (short) basis of $\Lambda$ that allows sampling module elements following a discrete Gaussian distribution defined over an arbitrary coset $\Lambda_t=\left\{(\boldsymbol{u}, \boldsymbol{v}) \in \mathcal{R}^2: \boldsymbol{u}+\boldsymbol{h} \boldsymbol{v}=\boldsymbol{t} \bmod q\right\}$.
	\item The signing algorithm first hashes $m$ to $y=t \in \mathcal{R}_q$, uses the secret basis to obtain a preimage $x=\left(\boldsymbol{s}_1, \boldsymbol{s}_2\right) \in \Lambda_t$, and outputs $\sigma=\left(\boldsymbol{s}_1, \boldsymbol{s}_2, r\right)$ as a signature. 
	\item The verification conditions are simply (1) $\boldsymbol{s}_1+\boldsymbol{h} \boldsymbol{s}_2=\mathrm{H}(r, m) \bmod q$, and (2) $\left\|\left(\boldsymbol{s}_1, \boldsymbol{s}_2\right)\right\|_2 \leq \beta \ll q$, where $\beta$ is determined by a Gaussian parameter.
\end{itemize}
Falcon stands out for its compactness, minimising both public key and signature sizes.
\end{frame}

\begin{frame}
\frametitle{What is LaBRADOR}
\begin{itemize}
	\item The LaBRADOR protocol is an \emph{iterative multi-round public-coin interactive proof} (Bulletproofs-style), which can be made non-interactive in the random oracle model by applying the Fiat-Shamir transform. The security of LaBRADOR relies on the hardness of \emph{the Module Shortest Integer Solution problem}.
	\item Native language (DPCS) well suited for Falcon verification
	\item \textbf{Lazarus} is a framework for implementing lattice-based zero-knowledge arguments in Rust. 
	\item In active developing, LaBRADOR PoC will be the very 1st milestone.
	\item \includegraphics[scale=0.2]{lazarus-repo.png}  pls scan and give us a star :)
\end{itemize}

\end{frame}

\begin{frame}
	\frametitle{LaCOM}
	\includegraphics[scale=0.3]{lacom.png}
	
	\small{We are NOT hiring, we BUDL for the future.}
\end{frame}

\begin{frame}
\frametitle{Aggregating Falcon Signatures with LaBRADOR}
\includegraphics[scale=0.3]{agg-falcon-paper.png}
\includegraphics[scale=0.3]{agg-f-l.png}
\end{frame}

\begin{frame}

\frametitle{(PQ) Aggregate Signatures}
\includegraphics[scale=0.3]{aggsig.png}

Aggregate BLS Signatures [BGLS`03]

Aggregate Schnorr Signatures [BN`06]

Multi/Threshold ECDSA [Lindell`17,GG`18,GG`20]

\end{frame}

\begin{frame}
\frametitle{Aggregate Signature: Correctness \& Security}
\begin{itemize}
	\item $\operatorname{AggSign}\left(\mathrm{pp},\left\{\mathrm{pk}_i, m_i, \sigma_i\right\}_{i \in[N]}\right) \rightarrow \sigma_{\text {agg }}$ 
	\item $\operatorname{AggVer}\left(\mathrm{pp},\left\{\mathrm{pk}_i, m_i\right\}_{i \in[N]}, \sigma_{\text {agg }}\right) \rightarrow b$
\end{itemize}

\begin{definition}[Correctness]
	For all $\lambda, N \in \mathbb{N}$ it yields

$$
\operatorname{Pr}\left[\operatorname{AggVer}\left(\left\{\mathrm{pk}_i, m_i\right\}_{i \in[N]}, \sigma_{\mathrm{agg}}\right)=1\right]=1-\operatorname{negl}(\lambda)
$$

where $\mathrm{pp} \leftarrow \operatorname{Setup}\left(1^\lambda\right), m_i \in M,\left(\mathrm{sk}_i, \mathrm{pk}_i\right) \leftarrow \operatorname{Gen}(\mathrm{pp}), \sigma_i \leftarrow \operatorname{Sign}\left(\mathrm{sk}_i, m_i\right)$ for all $i \in[N]$ and $\sigma_{\text {agg }} \leftarrow$ $\operatorname{AggSign}\left(\left\{\mathrm{pk}_i, m_i, \sigma_i\right\}_{i \in[N]}\right)$.
\end{definition}

	
\end{frame}

\begin{frame}
	\frametitle{Aggregate Signature: Correctness \& Security}
	
	\begin{definition}[Unforgeability]
	
		An AS scheme satisfies existential unforgeabilty in the aggregate chosen key model (EU-ACK), if for all PPT adversaries $\mathcal{A}$,
		$$
\operatorname{Adv}_{\mathrm{AS}}^{\mathrm{EU}-\operatorname{ACK}}(\mathcal{A}):= 
\operatorname{Pr}\left[{\left.\operatorname{EU}-\operatorname{ACK}_{\mathrm{AS}}(\mathcal{A}, \lambda)=1\right]=\operatorname{negl}(\lambda)}\right.
$$
	
	\end{definition}
\includegraphics[scale=0.45]{euack.png}	
\end{frame}



\begin{frame}
\frametitle{Computing hash locally}

\begin{quote}
	Batch-proving knowledge of $N$ Gaussian samples ( $\boldsymbol{s}_{i, 1}$, $\left.\boldsymbol{s}_{i, 2}\right)_{i \in[N]}$ and salts $\left(r_i\right)_{i \in[N]}$ meeting verification conditions w.r.t. a list $\left(m_i, \boldsymbol{h}_i\right)_{i \in[N]}$ of  messages and public keys, respectively. 
\end{quote} 


However, generating proof of correct hash computation is not only costly, but also leads to heuristic security guarantees: an aggregator may need a concrete description of $H$ as a hash function, while Falcon has only been proven secure if $H$ is modeled as a random oracle. And the size of salt $r_i$ is much smaller than ( $s_{i, 1}, s_{i, 2}$ )

\begin{itemize}
	\item Aggregator include salts $r_i$ in the aggregated signature 
	\item generate a proof for $t_i=s_{i, 1}+h_i s_{i, 2} \bmod q$ for a public statement $t_i$, 
	\item Verifier compute $\boldsymbol{t}_i=\mathrm{H}\left(r_i, m_i\right)$ locally. 
\end{itemize}

\end{frame}


\begin{frame}
	\frametitle{Adapting LaBRADOR for Aggregating Falcon Signatures}
	
	Principal relation of LaBRADOR:
	
	$$
f\left(\overrightarrow{\boldsymbol{w}}_1, \ldots, \overrightarrow{\boldsymbol{w}}_r\right):=\sum_{i, j=1}^r \boldsymbol{a}_{i, j}\left\langle\overrightarrow{\boldsymbol{w}}_i, \overrightarrow{\boldsymbol{w}}_j\right\rangle+\sum_{i=1}^r\left\langle\overrightarrow{\boldsymbol{\varphi}}_i, \overrightarrow{\boldsymbol{w}}_i\right\rangle-\boldsymbol{b}
$$

\begin{itemize}
	\item The verification equation and norm bound of a single Falcon signature seem quite compatible with the principal relation of LaBRADOR. 
	\item To aggregate $N$ signatures, one might then try to extend the statement to contain a verification equation for each signature and a combined norm bound: $\forall i=1, \ldots, N, \boldsymbol{s}_{i, 1}+\boldsymbol{h}_i \boldsymbol{s}_{i, 2}-\boldsymbol{t}_i=\mathbf{0} \bmod q$ and $\sum_{i=1}^N \sum_{j=1}^2\left\|\boldsymbol{s}_{i, j}\right\|_2^2 \leq N \beta^2$. 
\end{itemize}
If LaBRADOR was instantiated over the ring used by Falcon then $\left(s_{i, 1}, s_{i, 2}\right)_{i \in[N]}$ could be used directly as witness vector. However, there are several problems with this approach.
	
\end{frame}

\begin{frame}
	\frametitle{Adapting LaBRADOR for Aggregating Falcon Signatures}
	
	\begin{itemize}
		\item The norm check in LaBRADOR is both approximate and with respect to the entire witness. This introduces a degree of slack that grows with the number of signatures $N$. $\rightarrow$ Modify the first iteration of the LaBRADOR protocol to use the approach of [GHL22] for an exact proof of smallness.
		\item The modular Johnson-Lindenstrauss projections are used, which require that the norm bound $b$ of the statement satisfies the inequality $\sqrt{\lambda} b \leq q / C_1$ for security level $\lambda$ and some corresponding constant $C_1$. For both Falcon parameter sets this is not satisfied. $\rightarrow$ Reformulate the statement and witness so that the LaBRADOR protocol uses a separate modulus $q^{\prime}$, different from $q$
		\item the number of initial witness elements $r=2 N$ and their rank $n=1$ is quite unbalanced (bad for performance). $\rightarrow$ Present an alternative formulation of the constraints that achieves a better balance between these parameters. 
	\end{itemize}
	

	

\end{frame}



\begin{frame}
	\frametitle{Reducing proof size by moving to subrings}
	
	\begin{itemize}
		\item To significantly compress the proof sizes by using existing techniques from [LNPS21] to move to subrings $\mathcal{S}$ of smaller degrees $d^{\prime}$.
		\item Moving from the ring $\mathcal{R}$ of degree $d$ to a subring $\mathcal{S}$ of smaller degree $d^{\prime}=d / c$, which improves proof sizes.
		\item Reduces the proof sizes by at least a multiplicative factor 2 .
	\end{itemize}
\end{frame}

\begin{frame}
\frametitle{Choice of Ring and Challenge Space}

\begin{itemize}
	\item 2-splitting modulus $\rightarrow$ Lifted to composite modulus for NTT [CHK+'22]
	\item see more in Greyhound [NS`24]
\end{itemize}
	
\includegraphics[scale=0.3]{greyhound.png}
\end{frame}

\begin{frame}
\frametitle{Estimates and Comparison}

Single Falcon sig size $\approx$ 666 B


\includegraphics[scale=0.25]{size.png}

Concatenation $\approx$ 682 KB  , -82\% \footnote{\url{https://github.com/dfaranha/aggregate-falcon}}

\includegraphics[scale=0.4]{size2.png}

\end{frame}



\begin{frame}
	\centering \Large{Thanks!}
	
	\begin{columns}
	\column{0.5\textwidth}
	\centering
	\includegraphics[width=3cm]{zkpunk-qrcode.png}
    \column{0.5\textwidth}
    \centering
    \includegraphics[width=3cm]{lazarus-repo.png}
 \end{columns}
\end{frame}

\end{document}
